\documentclass[addpoints,9pt]{exam}
\usepackage{verbatim, multicol, tabularx,}
\usepackage{amsmath,amsthm, amssymb, latexsym, listings, qtree}

\lstset{frame=tb,
  language=Java,
  aboveskip=1mm,
  belowskip=0mm,
  showstringspaces=false,
  columns=flexible,
  basicstyle={\ttfamily},
  numbers=none,
  frame=single,
  breaklines=true,
  breakatwhitespace=true
}


\title{CS 2316 Exam 3}
\date{Practice}
\setcounter{page}{0}
\begin{document}

\maketitle
\thispagestyle{head}
%% \firstpageheader{}
%%               {\tiny Copyright \textcopyright\ 2016 All rights reserved. Duplication and/or usage for purposes of any kind without permission is strictly forbidden.}
%%               {}


\runningheader{}
              {\tiny Copyright \textcopyright\ 2017 All rights reserved. Duplication and/or usage for purposes of any kind without permission is strictly forbidden.}
              {}

\footer{Page \thepage\ of \numpages}
              {}
              {Points available: \pointsonpage{\thepage} -
               points lost: \makebox[.5in]{\hrulefill} =
               points earned:  \makebox[.5in]{\hrulefill}.
              Graded by: \makebox[.5in]{\hrulefill}}


\ifprintanswers
\begin{center}
{\LARGE ANSWER KEY}
\end{center}
\else
\vspace{0.1in}
\hbox to \textwidth{Name (print clearly): \enspace\hrulefill}
\vspace{0.3in}
\hbox to \textwidth{T-Square ID (gtg, gth, msmith3, etc): \underline{\hspace{2in}} Section (e.g., B1): \underline{\hspace{.75in}}}
\vspace{0.3in}
\hbox to \textwidth{Signature: \enspace\hrulefill}

\fi

\vfill

\begin{itemize}
\item Failure to properly fill in the information on this page will result in a deduction of up to 5 points from your exam score.
\item Signing signifies you are aware of and in accordance with the {\bf Academic Honor Code of Georgia Tech} and that you will not discuss this exam with other students.
\item Calculators and cell phones are NOT allowed.
\item Answers containing Python code must use valid Python code, including case-sensitivity, syntax, and API correctness.
\end{itemize}

\vfill

% Points Table
%\begin{center}
\addpoints
%\gradetable[v][pages]
%\end{center}

% Points Table
\begin{center}
\renewcommand{\arraystretch}{2}
\begin{tabularx}{\textwidth}{|l|c|X|X|X|}
        \hline
Question & Points per Page & Points Lost & Points Earned & Graded By \\
\hline
Page 1 & \pointsonpage{1} & - & =  &\\
\hline
Page 2 & \pointsonpage{2} & - & =  &\\
\hline
Page 3 & \pointsonpage{3} & - & =  &\\
\hline
Page 4 & \pointsonpage{4} & - & =  &\\
\hline
Page 5 & \pointsonpage{5} & - & =  &\\
\hline
TOTAL & \numpoints & - & =  & \\
\hline
\end{tabularx}
\end{center}

\newpage

%\normalsize

\pointsinmargin
\bracketedpoints

\marginpointname{}
%%%%%%%%%%%%%%%%%%%%%%%%%%%%%%%%%%%%%%%%%%%%%%%%%%%%%%%%%%%%%%%%%%%%%%%%%%%%

\begin{questions}

\question {\bf Multiple Choice} Circle the letter of the best answer.

\begin{parts}

\part[3] Given this definition:

\begin{lstlisting}[language=Python]
d = {
  "people": {
    "person": [
      {
        "firstName": "Alan",
        "lastName": "Turing",
        "professions": {
          "profession": ["Computer Scientist", "Mathematician",
                         "Computer Scientist", "Cryptographer"]
         }
       },
       {
         "firstName": "Stephen",
         "lastName": "Hawking",
         "professions": {
           "profession": ["Physicist", "Comedian"]
         }
       }
    ]
  }
}
\end{lstlisting}

\part[3] Which of the following returns the second profession of Stephen Hawking (whose value would be {\tt 'Comedian'})?

\begin{choices}
  \correctchoice {\tt d['people']['person'][1]['professions']['profession'][1]}
  \choice {\tt d['people']['person'][1]['professions']['profession']}
  \choice {\tt d['people']['person'][1]['professions']['Comedian']}
\end{choices}

\part[3] What's the type of {\tt d['people']['person'][1]['professions']['profession']}

\begin{choices}
  \choice {\tt tuple}
  \choice {\tt dict}
  \correctchoice {\tt list}
\end{choices}

\part[3] What's the value of {\tt d['people']['person'][0]['firstName']}?

\begin{choices}
  \choice 'Hawking'
  \choice 'Stephen'
  \choice 'Turing'
  \correctchoice 'Alan'
\end{choices}

\part[3] Which of the following Python expressions opens a file for reading as text?

\begin{choices}
\choice {\tt open "season"}
\choice {\tt open("borders", 'wb')}
\correctchoice {\tt open("sesame", 'r')}
\choice All of the above
\end{choices}

\end{parts}

\newpage

\question  {\bf Multiple Choice} Circle the letter of the best choice.

\begin{parts}

\part[3] The fundamental data abstraction in relational databases is the table.

\begin{choices}
\correctchoice True
\choice False
\end{choices}

\part[3] In order for a foreign key in one table to reference a primary key in another table, it must have the same name.

\begin{choices}
  \choice True
  \correctchoice False
\end{choices}

\part[3] An author can write many books and a book can have many authors.  What kind of cardinality relationship exists between authors and books?

\begin{choices}
\correctchoice many to many
\choice one to one
\choice one to many
\end{choices}

\part[3] The CSV data model can encode any data model that the XML data model can.

\begin{choices}
\choice True
\correctchoice False
\end{choices}

\part[3] Which of the following is {\bf not} well-formed XML?

\begin{choices}
\choice {\tt <a> <b> c </b> </a>}
\correctchoice {\tt <a> <b> <c> </b> </a>}
\choice {\tt <a> <b> <c/> </b> </a>}
\choice {\tt <a> <b> <c> d </c> </b> </a>}
\end{choices}


\end{parts}

\newpage

\question {\bf Short Answer}

\begin{parts}

\part[5] What command would you type in iPython to find your present working directory?

\ifprintanswers
\begin{lstlisting}[language=Python]
pwd or %pwd
\end{lstlisting}
\else
\vspace{1in}
\fi

\part[5] How would you find out what the {\tt \%prun} command does in iPython?

\ifprintanswers
\begin{lstlisting}[language=Python]
%prun?
\end{lstlisting}
\else
\vspace{1in}
\fi

\part[5] Write an expression that creates a NumPy array of 5 integers. Assume {\tt import numpy as np} has been done.

\ifprintanswers
Many possibilities
\begin{lstlisting}[language=Python]
np.arange(5)
\end{lstlisting}

\begin{lstlisting}[language=Python]
np.zeros(5, dtype=int) # or np.ones
\end{lstlisting}

\begin{lstlisting}[language=Python]
np.array([0, 1, 2, 3, 4])
\end{lstlisting}

\else
\vspace{1in}
\fi

\part[5]  Write an expression that creates a 3 x 3  NumPy array of integers. Assume {\tt import numpy as np} has been done.

\ifprintanswers
Many possibilities
\begin{lstlisting}[language=Python]
np.arange(9).reshape((3,3))
\end{lstlisting}

\begin{lstlisting}[language=Python]
np.zeros(9, dtype=int).reshape((3,3))
\end{lstlisting}

\begin{lstlisting}[language=Python]
np.array([[1,2,3],[4,5,6],[7,8,9]])
\end{lstlisting}

\else
\vspace{1in}
\fi

\part[5] Given a dictionary {\tt d} created by {\tt d = dict(zip(['a', 'b', 'c', 'd'], range(4)))}, write a statement that creates a Pandas Series from {\tt d} and assigns it to the variable {\tt data}. Assume {\tt import pandas as pd} has been done.

\ifprintanswers
Many possibilities
\begin{lstlisting}[language=Python]
data = pd.Series(d)
\end{lstlisting}
\else
\vspace{1in}
\fi

\part[5] After creating the series {\tt data} above, what would {\tt data['b']} return?
\ifprintanswers
\begin{lstlisting}[language=Python]
1
\end{lstlisting}
\else
\vspace{1in}
\fi


\end{parts}

\newpage

\question Short answer

Given:
\begin{lstlisting}[language=Python]
salary = {"Data Scientist": 110000,
          "DevOps Engineer": 110000,
          "Data Engineer": 106000,
          "Analytics Manager": 112000,
          "Database Administrator": 93000,
          "Software Architect": 125000,
          "Software Engineer": 101000,
          "Supply Chain Manager": 100000}
openings = {"Data Scientist": 4184,
            "DevOps Engineer": 2725,
            "Data Engineer": 2599,
            "Analytics Manager": 1958,
            "Database Administrator": 2877,
            "Software Architect": 2232,
            "Software Engineer": 17085,
            "Supply Chain Manager": 1270}
\end{lstlisting}

\begin{parts}

\part[5] Write a statement that assigns to {\tt salary\_data} a Panda series with the data from the {\tt salary} dictionary.

\ifprintanswers
\begin{verbatim}
salary_data = pd.Series(salary)
\end{verbatim}
\else
\vspace{.75in}
\fi

\part[5] After the assignment above, what is the value of {\tt salary\_data[’Software Engineer’]}

\ifprintanswers
\begin{verbatim}
101000
\end{verbatim}
\else
\vspace{.75in}
\fi

\part[5] Write a statement that assigns to {\tt jobs} a Panda DataFrame from the data in the {\tt salary} and {\tt openings} dictionaries with {\tt 'salary'} as the heading for the salary column and {\tt 'openings'} as the heading for the openings column.

\ifprintanswers
\begin{verbatim}
1 pt   1 pt         1 pt                    1 pt (1 pt remainder of syntax)
jobs = pd.DataFrame({'salary': salary_data, 'openings': openings})
\end{verbatim}
\else
\vspace{.75in}
\fi

\part[5] Write an expression that returns all the jobs in the {\tt jobs} DataFrame with salary greater than 100000.

\ifprintanswers
\begin{verbatim}
 jobs[jobs[’salary’] > 100000]
\end{verbatim}
\else
\vspace{.75in}
\fi

\part[5] Write an assignment statement that adds a column to {\tt jobs} called {\tt '6 figures'} whose values are {\tt True} for jobs with salaries greater than 100000 and {\tt False} otherwise.

\ifprintanswers
\begin{verbatim}
2 pts               3 pts
jobs[’6 Figures’] = jobs[’salary’] > 100000
\end{verbatim}
\else
\vspace{.75in}
\fi


\end{parts}

\newpage

\question {\bf Short Answer}

\begin{parts}

Assuming a database with the following schema is stored in an SQLite3 database file named {\tt dorms.db},

\begin{lstlisting}[language=SQL]
create table dorm (
    dorm_id integer primary key autoincrement,
    name text,
    spaces integer
);
create table stud (
    stud_id integer primary key autoincrement,
    name text,
    gpa float,
    dorm_id integer references dorm(dorm_id)
);
\end{lstlisting}

\part[15] write a snippet of Python code that queries the database and stores in a variable named {\tt dorm\_assignments} a list whose elements are tuples, where each tuple contains a student name and the name of the dorm that student lives in, e.g., tuples like {\tt ('Cartman', 'Armstrong')}. Assume the {\tt sqlite3} module is imported.

\begin{solution}[2in]
\begin{verbatim}
conn = sqlite3.connect("dorms.db")                    # 2 pts
curs = conn.cursor()                                  # 1 pt
curs.execute("select stud.name, dorm.name " +         # 2 pts execute(), 3 pts select header
             "from student join dorm using(dorm_id)") # 4 pts correct join
dorm_assignments = curs.fetchall()                    # 3 pts (other methods also correct)
\end{verbatim}
\end{solution}

\part[5] Write a single Python expression that creates a tuple mapping student names to the names of the dorms they live in using the {\tt dorm\_assignments} list created above.

\begin{solution}[1in]
\begin{verbatim}
dict(dorm_assignments)
or
 2 pts         2 pts         1 pt
{stud:dorm for stud, dorm in dorm_assignments}
or
{t[0]:t[1] for t in dorm_assignments}
\end{verbatim}
\end{solution}


\part[5] Write a single Python expressions that creates a list of students in Armstrong  using the {\tt dorm\_assignments} list created above.

\begin{solution}[1in]
\begin{verbatim}
 1 pt     1 pt         1 pt             2 pts
[stud for stud,dorm in dorm_assignments if dorm == 'Armstrong']
or
[t[0] for t in dorm_assignments if t[1] == 'Armstrong']
\end{verbatim}
\end{solution}


\end{parts}

\end{questions}

\end{document}
